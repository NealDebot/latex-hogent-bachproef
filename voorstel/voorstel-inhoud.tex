%---------- Inleiding ---------------------------------------------------------

% TODO: Is dit voorstel gebaseerd op een paper van Research Methods die je
% vorig jaar hebt ingediend? Heb je daarbij eventueel samengewerkt met een
% andere student?
% Zo ja, haal dan de tekst hieronder uit commentaar en pas aan.

%\paragraph{Opmerking}

% Dit voorstel is gebaseerd op het onderzoeksvoorstel dat werd geschreven in het
% kader van het vak Research Methods dat ik (vorig/dit) academiejaar heb
% uitgewerkt (met medesturent VOORNAAM NAAM als mede-auteur).
% 

\section{Inleiding}%
\label{sec:inleiding}

In de digitale wereld van vandaag is er een groeiende vraag naar mobiele en webapplicaties, zowel binnen grote organisaties als bij kleinere ondernemingen.
Applicaties bieden bedrijven de mogelijkheid om hun klanten een gepersonaliseerde ervaring te geven en een sterke merkidentiteit te creëren. 
Het ontwikkelen van een volledig nieuwe, op maat gemaakte applicatie kan echter hoge kosten en een lange doorlooptijd met zich meebrengen. 
Om die redenen zoeken steeds meer bedrijven naar alternatieven die kosteneffectief en snel te implementeren zijn.
Whitelabel apps, standaardapplicaties die aangepast kunnen worden aan de huisstijl van een bedrijf, bieden een aantrekkelijk alternatief. 
Dit onderzoek richt zich op het fenomeen van whitelabel apps en de impact ervan op de kosten en de beschikbare implementatiemogelijkheden.\\

Dit onderzoek richt zich in eerste instantie op bedrijven en organisaties die overwegen om een applicatie te lanceren of hun huidige applicatie te vervangen, maar beperkt budget of middelen hebben voor volledige maatwerkontwikkeling.\\
Daarnaast is dit onderzoek relevant voor softwareontwikkelingsbedrijven die hun klanten whitelabeloplossingen willen aanbieden en voor productmanagers en IT-consultants die verantwoordelijk zijn voor de digitale strategie binnen bedrijven.\\

Hoewel whitelabel apps aantrekkelijk zijn door hun lagere kosten en snellere implementatie, is er nog weinig onderzoek gedaan naar de werkelijke kostenvoordelen van whitelabeling in vergelijking met maatwerkontwikkelingen. 
Daarnaast zijn er diverse strategieën en aanpassingsopties binnen whitelabeling, maar er ontbreekt overzicht en inzicht in de beste keuzes per bedrijfstype en sector. 
Dit gebrek aan informatie kan leiden tot verkeerde investeringen of inefficiënte implementaties, wat bedrijven belemmert in hun doel om effectief en kostenefficiënt te digitaliseren.\\

Om de probleemstelling te verkennen, worden de volgende onderzoeksvragen gesteld:
\begin{enumerate}
  \item Welk effect heeft de ontwikkeling van whitelabel apps op de kosten van applicatieontwikkeling in vergelijking met maatwerk?
  \item Welke verschillende mogelijkheden bestaan er voor whitelabeling, en wat zijn hun voodelen en nadelen?
\end{enumerate}

De doelstelling van dit onderzoek is om inzicht te verschaffen in de kostenbesparingen die whitelabel apps kunnen opleveren en om een overzicht te geven van de verschillende implementatieopties die beschikbaar zijn. 
Door het beantwoorden van de onderzoeksvragen zal dit onderzoek bedrijven helpen een geïnformeerde beslissing te maken over het al dan niet inzetten van whitelabel apps in hun digitale strategie. 
Uiteindelijk beoogt het onderzoek bij te dragen aan een efficiëntere en effectievere inzet van whitelabel apps binnen het bedrijfsleven.

%---------- Stand van zaken ---------------------------------------------------

\section{Literatuurstudie}%
\label{sec:literatuurstudie}

% Voor literatuurverwijzingen zijn er twee belangrijke commando's:
% \autocite{KEY} => (Auteur, jaartal) Gebruik dit als de naam van de auteur
%   geen onderdeel is van de zin.
% \textcite{KEY} => Auteur (jaartal)  Gebruik dit als de auteursnaam wel een
%   functie heeft in de zin (bv. ``Uit onderzoek door Doll & Hill (1954) bleek
%   ...'')

\subsection{Wat is een Whitelabel App?}
Een whitelabel app is een applicatie die door een ontwikkelaar is gemaakt met de bedoeling dat deze door andere bedrijven kan worden aangepast en onder hun eigen merknaam kan worden aangeboden.
Het concept van whitelabeling is niet nieuw en wordt al langer toegepast in diverse industrieën, zoals in de voedingsmiddelen- en elektronicasector, waar producten door verschillende bedrijven onder hun eigen merk verkocht worden. 
Bij whitelabel apps geldt een vergelijkbare aanpak, waarbij een gestandaardiseerde basisapplicatie wordt ontwikkeld die qua functionaliteiten generiek en flexibel genoeg is om aan te passen aan de branding- en stijlrichtlijnen van individuele bedrijven \autocite{Candelario2024}.

In technische zin zijn whitelabel apps vaak opgebouwd uit herbruikbare componenten en modulaire structuren, wat het proces van aanpassing vergemakkelijkt. 
Dit maakt het voor bedrijven aantrekkelijker, omdat de aanpassing van een whitelabel app doorgaans minder tijd en geld kost dan het ontwikkelen van een app van nul af aan \autocite{Vendesta2019}.

\subsection{Voordelen van Whitelabel Apps}
Whitelabel apps bieden diverse voordelen die bijdragen aan hun populariteit, met name bij kleine en middelgrote ondernemingen. 
Ten eerste zijn de kosten doorgaans lager dan die voor maatwerkapplicaties, wat een groot voordeel is voor bedrijven met beperkte middelen. 
Daarnaast is de time-to-market korter, omdat de kern van de applicatie al ontwikkeld is en alleen nog moet worden aangepast aan de wensen van de klant \autocite{Struk2023}.

Een ander voordeel is de technische ondersteuning die vaak wordt aangeboden door de originele ontwikkelaar van de whitelabel oplossing. 
Dit betekent dat updates en beveiligingspatches centraal beheerd kunnen worden, waardoor bedrijven niet zelf de technische kennis hoeven te hebben of personeel hoeven in te schakelen om het onderhoud te verzorgen. 
Bovendien zorgt het gebruik van een bewezen framework of basisstructuur ervoor dat whitelabel apps doorgaans stabieler zijn en minder bugs bevatten \autocite{Candelario2024}.

\subsection{Nadelen van Whitelabel Apps}
Ondanks de voordelen zijn er ook verschillende nadelen verbonden aan whitelabel apps. 
Een veelvoorkomend nadeel is de beperkte flexibiliteit in functionaliteit en ontwerp. 
Omdat whitelabel apps meestal gebouwd zijn op een standaardtemplate, zijn de mogelijkheden om specifieke functionaliteiten toe te voegen of bestaande functies te wijzigen vaak beperkt. 
Dit kan ertoe leiden dat bedrijven concessies moeten doen op het gebied van personalisatie en unieke functies \autocite{Vendesta2023}.

Daarnaast kan het gebruik van whitelabel apps de merkidentiteit van een bedrijf schaden. 
Omdat dezelfde basisapplicatie door meerdere bedrijven gebruikt kan worden, bestaat het risico dat de applicatie niet uniek aanvoelt en dat gebruikers de associatie met de specifieke merkidentiteit van een bedrijf verliezen. 
Ten slotte kunnen sommige bedrijven die afhankelijk zijn van whitelabel oplossingen het risico lopen dat de ontwikkelaar stopt met de ondersteuning of updates, wat gevolgen kan hebben voor de beveiliging en werking van de app \autocite{Vendesta2023}.

\subsection{Huidig Onderzoek naar Whitelabel Apps}
Hoewel whitelabeling een bekend concept is, is er relatief weinig academisch onderzoek gedaan naar de specifieke effecten van whitelabel apps op kosten en functionaliteit. 
Het meeste onderzoek is gericht op de algemene voordelen en nadelen van whitelabeling in diverse sectoren, zoals de detailhandel en financiële dienstverlening. 
Daarnaast bestaan er enkele studies over de kosteneffectiviteit van whitelabel producten, maar specifieke data over apps is schaars en vaak gericht op de technische aspecten van softwareontwikkeling. 
Dit gebrek aan diepgaande studies maakt dit onderzoeksvoorstel relevant, omdat het inzicht wil bieden in de kosten-batenanalyse en implementatiemogelijkheden van whitelabel apps voor bedrijven die digitale toepassingen overwegen.
%---------- Methodologie ------------------------------------------------------
\section{Methodologie}%
\label{sec:methodologie}

Hier beschrijf je hoe je van plan bent het onderzoek te voeren. Welke onderzoekstechniek ga je toepassen om elk van je onderzoeksvragen te beantwoorden? Gebruik je hiervoor literatuurstudie, interviews met belanghebbenden (bv.~voor requirements-analyse), experimenten, simulaties, vergelijkende studie, risico-analyse, PoC, \ldots?

Valt je onderwerp onder één van de typische soorten bachelorproeven die besproken zijn in de lessen Research Methods (bv.\ vergelijkende studie of risico-analyse)? Zorg er dan ook voor dat we duidelijk de verschillende stappen terug vinden die we verwachten in dit soort onderzoek!

Vermijd onderzoekstechnieken die geen objectieve, meetbare resultaten kunnen opleveren. Enquêtes, bijvoorbeeld, zijn voor een bachelorproef informatica meestal \textbf{niet geschikt}. De antwoorden zijn eerder meningen dan feiten en in de praktijk blijkt het ook bijzonder moeilijk om voldoende respondenten te vinden. Studenten die een enquête willen voeren, hebben meestal ook geen goede definitie van de populatie, waardoor ook niet kan aangetoond worden dat eventuele resultaten representatief zijn.

Uit dit onderdeel moet duidelijk naar voor komen dat je bachelorproef ook technisch voldoen\-de diepgang zal bevatten. Het zou niet kloppen als een bachelorproef informatica ook door bv.\ een student marketing zou kunnen uitgevoerd worden.

Je beschrijft ook al welke tools (hardware, software, diensten, \ldots) je denkt hiervoor te gebruiken of te ontwikkelen.

Probeer ook een tijdschatting te maken. Hoe lang zal je met elke fase van je onderzoek bezig zijn en wat zijn de concrete \emph{deliverables} in elke fase?

%---------- Verwachte resultaten ----------------------------------------------
\section{Verwacht resultaat, conclusie}%
\label{sec:verwachte_resultaten}

Hier beschrijf je welke resultaten je verwacht. Als je metingen en simulaties uitvoert, kan je hier al mock-ups maken van de grafieken samen met de verwachte conclusies. Benoem zeker al je assen en de onderdelen van de grafiek die je gaat gebruiken. Dit zorgt ervoor dat je concreet weet welk soort data je moet verzamelen en hoe je die moet meten.

Wat heeft de doelgroep van je onderzoek aan het resultaat? Op welke manier zorgt jouw bachelorproef voor een meerwaarde?

Hier beschrijf je wat je verwacht uit je onderzoek, met de motivatie waarom. Het is \textbf{niet} erg indien uit je onderzoek andere resultaten en conclusies vloeien dan dat je hier beschrijft: het is dan juist interessant om te onderzoeken waarom jouw hypothesen niet overeenkomen met de resultaten.

